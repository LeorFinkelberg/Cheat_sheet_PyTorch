\documentclass[%
	11pt,
	a4paper,
	utf8,
	%twocolumn
		]{article}	

\usepackage{style_packages/podvoyskiy_article_extended}


\begin{document}
\title{Приемы работы с библиотекой PyTorch}

\author{}

\date{}
\maketitle

\thispagestyle{fancy}

\tableofcontents

\section{Вводные замечания}

Чаще всего цикл обучения модели реализуют в виде обычного цикла \verb|for| Python.




% Источники в "Газовой промышленности" нумеруются по мере упоминания 
\begin{thebibliography}{99}\addcontentsline{toc}{section}{Список литературы}
	\bibitem{beazley:python-2010}{\emph{Бизли Д.} Python. Подробный справочник. -- СПб.: Символ-Плюс, 2010. -- 864 с.}
	
	\bibitem{beazley:python_cookbook-2019}{\emph{Бизли Д.} Python. Книга рецептов. -- М.: ДМК Пресс., 2019. -- 648 с.}

    \bibitem{mckinney:pandas-2015}{\emph{Маккинли У.} Python и анализ данных, 2015. -- 482 с.}
	
	\bibitem{ramalho:python-2022}{\emph{Рамальо Л.} Python -- к вершинам мастерства: Лаконичное и эффективное программирование. -- М.: МК Пресс, 2022. -- 898 с.}
	
	\bibitem{heydt:pandas-2019}{\emph{Хейдт М., Груздев А.} Изучаем pandas. -- М.: ДМК Пресс, 2019. -- 682 с.}
	
	\bibitem{hostmann:scala-2013}{\emph{Хостманн К.} Scala для нетерпеливых. -- М.: ДМК Пресс, 2013. -- 408 с.}
\end{thebibliography}

%\listoffigures\addcontentsline{toc}{section}{Список иллюстраций}

%\lstlistoflistings\addcontentsline{toc}{section}{Список листингов}

\end{document}
